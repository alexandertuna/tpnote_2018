\section{Introduction}
\label{sec:intro}

 The Micromegas trigger processor (MMTP) should provide muon track candidates to the muon endcap L1 trigger
 with an expected 1 (20) mrad accuracy in the polar (azimuthal) angle.
 The MMTP algorithm and its evolution over time are described in detail in Refs.~\cite{nswtdr,brian,steve}.
 The first stage of the trigger algorithm, referred to as the finder, 
 converts Micromegas hits into polar slopes assuming tracks originate from the interaction point.

 An octuplet wedge is divided into N slope roads. The size of each slope road is 0.009 mrad or 16 Micromegas
 strips. Slope values are stored in a circular buffer defined as (N roads) $\times$ (8 planes) $\times$  (NBC), where
 NBC is the buffer time depth expressed in number of bunch crosses. In the following, the number $n$ of  planes used in a road
 is sometimes referred to as the $n$-level of a road.
 Track candidates require a minimum of 3 X and 3 U,V hits in a road within 2 BCs. 

 To avoid edge effects and because U,V panels are slanted by 1.5$\deg$, the search
 for a trigger in a given X road uses also the hits in the 2 neighbor
 roads, and that on the U,V roads also uses the hits in the 4 neighbor roads.
 Of course, this is not enough to cover efficiently the real detector - the longest strips require 8 neighbor roads -
 but it works for our reduced size Micromegas
 octuplet~\cite{noisy,noiseless}.

 If a candidate is found, the next algorithm stage, referred to as the fitter, calculates a global slope $\theta_{g}$ as the
 average of the hit slopes. At the same time, it fits with a straight line the slope of each hit as a
 function of its distance from the interaction point
 to determine a local slope $\theta_l$.
 A $\pm 15$ mrad cut is applied on the difference $\theta_g-\theta_l$  to
 improve background rejection.
 By using a combination of the X, U, and V slopes the fitter also provides the $\phi$ information.
 
This MMTP configuration has been tested with cosmic muons, and results are reported in Ref. \cite{mmtp}.
 The first conclusion of this study is that the time window used to form a trigger needs to be set to $6 - 8$ BCs
depending on the efficiency of each Micromegas plane. In other words, if each Micromegas plane were 100\% efficient,
a 98\% efficient trigger could be formed in a 6 BC window. This window should be 8 BC wide if the
Micromegas efficiency is in the ballpark of 90\%. This larger than expected BC window greatly increases the number
of triggers produced by random background hits. 

 In about 15\% of the events, a muon track is accompanied by a $\delta$-ray  with energy above a few keV.
 The data reminded us that hits due to $\delta$-rays  can replace some of the track hits
 in a given road. As a consequence of this irreducible background, the  $\theta_g-\theta_l$ cut results
 in a  $\simeq 4$\% inefficiency.
 It seems likely that the higher probability of accidental hits
 in the HL-LHC environment will increase this type of inefficiency and further spoil the required
 angular resolutions.

To quantify these effects, we developed a software 
 simulation of the MMTP algorithm and a parameterized Micromegas response based on data.
 This tool is referred to as the Harvard Octuplet Trigger Performance Optimization Tool (HOTPOT). 

