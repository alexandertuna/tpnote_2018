\section{Introduction}
\label{sec:intro}
The performance of the original Micromegas trigger processor algorithm has been implemented and tested with cosmic muons in Ref. \cite{mmtp}. We proposed and simulated the perfomance of an updated algorithm version, with the results described in Ref. ~\cite{tpsim}. The updated algorithm is robust against the large background rate expected at the High Luminosity LHC (HL-LHC). 
\par In the following results, we implement the proposed algorithm in firmware and test its performance with cosmic muons $\ge 0.8$~GeV$/c^2$. Over half a million events were recorded with the setup described in Ref. ~\cite{noisy,noiseless}. The setup includes a cosmic ray test stand consisting of an Micromegas octuplet, scintillators, modified front-end boards, and a trigger chain, with the minimal noise conditions described in ~\cite{utpc}. We use a VMM gain of 9 mV/fC and a peaktime of 200 ns. The ART signal is generated at threshold.
\par The ART signal is packaged by two FPGA-based ART Data Drive Cards (ADDC) V1s and sent to a VC707 FPGA evaluation board which has an implementation of the trigger processor algorithm. The scintillator trigger, which signals readout for the MMFE8 cards, is sent to the VC707. The arrival time of the scintillator trigger is recorded using a 640 MHz clock. More details of the hardware and data streams are discussed in ~\cite{mmtp}. 
\par The resulting trigger efficiency measured in the cosmic data is described.

